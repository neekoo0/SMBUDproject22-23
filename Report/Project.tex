% A LaTeX template for MSc Thesis submissions to 
% Politecnico di Milano (PoliMi) - School of Industrial and Information Engineering
%
% S. Bonetti, A. Gruttadauria, G. Mescolini, A. Zingaro
% e-mail: template-tesi-ingind@polimi.it
%
% Last Revision: October 2021
%
% Copyright 2021 Politecnico di Milano, Italy. NC-BY

\documentclass{Configuration_Files/PoliMi3i_thesis}

%------------------------------------------------------------------------------
%	REQUIRED PACKAGES AND  CONFIGURATIONS
%------------------------------------------------------------------------------

% CONFIGURATIONS
\usepackage{parskip} % For paragraph layout
\usepackage{setspace} % For using single or double spacing
\usepackage{emptypage} % To insert empty pages
\usepackage{multicol} % To write in multiple columns (executive summary)
\setlength\columnsep{15pt} % Column separation in executive summary
\setlength\parindent{0pt} % Indentation
\raggedbottom  

% PACKAGES FOR TITLES
\usepackage{titlesec}
% \titlespacing{\section}{left spacing}{before spacing}{after spacing}
\titlespacing{\section}{0pt}{3.3ex}{2ex}
\titlespacing{\subsection}{0pt}{3.3ex}{1.65ex}
\titlespacing{\subsubsection}{0pt}{3.3ex}{1ex}
\usepackage{color}

% PACKAGES FOR LANGUAGE AND FONT
\usepackage[english]{babel} % The document is in English  
\usepackage[utf8]{inputenc} % UTF8 encoding
\usepackage[T1]{fontenc} % Font encoding
\usepackage[11pt]{moresize} % Big fonts

% PACKAGES FOR IMAGES
\usepackage{graphicx}
\usepackage{transparent} % Enables transparent images
\usepackage{eso-pic} % For the background picture on the title page
\usepackage{subfig} % Numbered and caption subfigures using \subfloat.
\usepackage{tikz} % A package for high-quality hand-made figures.
\usetikzlibrary{}
\graphicspath{{./Images/}} % Directory of the images
\usepackage{caption} % Coloured captions
\usepackage{xcolor} % Coloured captions
\usepackage{amsthm,thmtools,xcolor} % Coloured "Theorem"
\usepackage{float}

% STANDARD MATH PACKAGES
\usepackage{amsmath}
\usepackage{amsthm}
\usepackage{amssymb}
\usepackage{amsfonts}
\usepackage{bm}
\usepackage[overload]{empheq} % For braced-style systems of equations.
\usepackage{fix-cm} % To override original LaTeX restrictions on sizes

% PACKAGES FOR TABLES
\usepackage{tabularx}
\usepackage{longtable} % Tables that can span several pages
\usepackage{colortbl}

% PACKAGES FOR ALGORITHMS (PSEUDO-CODE)
\usepackage{algorithm}
\usepackage{algorithmic}

% PACKAGES FOR REFERENCES & BIBLIOGRAPHY
\usepackage[colorlinks=true,linkcolor=black,anchorcolor=black,citecolor=black,filecolor=black,menucolor=black,runcolor=black,urlcolor=black]{hyperref} % Adds clickable links at references
\usepackage{cleveref}
\usepackage[square, numbers, sort&compress]{natbib} % Square brackets, citing references with numbers, citations sorted by appearance in the text and compressed
\bibliographystyle{abbrvnat} % You may use a different style adapted to your field

% OTHER PACKAGES
\usepackage{pdfpages} % To include a pdf file
\usepackage{afterpage}
\usepackage{lipsum} % DUMMY PACKAGE
\usepackage{fancyhdr} % For the headers
\fancyhf{}

% Input of configuration file. Do not change config.tex file unless you really know what you are doing. 
\input{Configuration_Files/config}

%----------------------------------------------------------------------------
%	NEW COMMANDS DEFINED
%----------------------------------------------------------------------------

% EXAMPLES OF NEW COMMANDS
\newcommand{\bea}{\begin{eqnarray}} % Shortcut for equation arrays
\newcommand{\eea}{\end{eqnarray}}
\newcommand{\e}[1]{\times 10^{#1}}  % Powers of 10 notation

%----------------------------------------------------------------------------
%	ADD YOUR PACKAGES (be careful of package interaction)
%----------------------------------------------------------------------------

%----------------------------------------------------------------------------
%	ADD YOUR DEFINITIONS AND COMMANDS (be careful of existing commands)
%----------------------------------------------------------------------------

%----------------------------------------------------------------------------
%	BEGIN OF YOUR DOCUMENT
%----------------------------------------------------------------------------

\begin{document}

\fancypagestyle{plain}{%
\fancyhf{} % Clear all header and footer fields
\fancyhead[RO,RE]{\thepage} %RO=right odd, RE=right even
\renewcommand{\headrulewidth}{0pt}
\renewcommand{\footrulewidth}{0pt}}

%----------------------------------------------------------------------------
%	TITLE PAGE
%----------------------------------------------------------------------------

\pagestyle{empty} % No page numbers
\frontmatter % Use roman page numbering style (i, ii, iii, iv...) for the preamble pages

\puttitle{
	title=Systems and Methods for Big and Unstructured Data Project,
	name1=Gabriele Ginestroni, % Author Name and Surname
	name2=Giacomo Gumiero,
	name3=Lorenzo Iovine,
	name4=Nicola Landini,
	name5=Francesco Leone,
	academicyear=2022-2023,
	groupnumber=10
} % These info will be put into your Title page 

%----------------------------------------------------------------------------
%	PREAMBLE PAGES: ABSTRACT (inglese e italiano), EXECUTIVE SUMMARY
%----------------------------------------------------------------------------
\startpreamble
\setcounter{page}{1} % Set page counter to 1

%----------------------------------------------------------------------------
%	LIST OF CONTENTS/FIGURES/TABLES/SYMBOLS
%----------------------------------------------------------------------------

% TABLE OF CONTENTS
\thispagestyle{empty}
\tableofcontents % Table of contents 
\thispagestyle{empty}
\cleardoublepage

%-------------------------------------------------------------------------
%	THESIS MAIN TEXT
%-------------------------------------------------------------------------
% In the main text of your thesis you can write the chapters in two different ways:
%
%(1) As presented in this template you can write:
%    \chapter{Title of the chapter}
%    *body of the chapter*
%
%(2) You can write your chapter in a separated .tex file and then include it in the main file with the following command:
%    \chapter{Title of the chapter}
%    \input{chapter_file.tex}
%
% Especially for long thesis, we recommend you the second option.

\addtocontents{toc}{\vspace{2em}} % Add a gap in the Contents, for aesthetics
\mainmatter % Begin numeric (1,2,3...) page numbering

\chapter{Introduction}
\label{ch:introduction}%
% The \label{...}% enables to remove the small indentation that is generated, always leave the % symbol.

In this chapter will be presented the problem specification and the hypothesis under which the database is implemented.

\section{Problem Specification}
\label{sec:prob_specs}
This project aims to build an Information System that handles scientific articles contained in the DBLP
bibliography. The project involves managing the type of every article and the associated DOI (Digital Object Identifier),
which identifies an article or a document and links to it on the web. Other entities to deal with are authors, identified by
an ID or ORCID (Open Researcher and Contributor ID), and their affiliations with organizations. In order to address the
problem, we will store data in a graph database, allowing us to visualize relations and handle information correctly.


\section{Equations}
\label{sec:eqs}
This section gives some examples of writing mathematical equations in your thesis.

Maxwell's equations read:
\begin{subequations}
    \label{eq:maxwell}
    \begin{align}[left=\empheqlbrace]
    \nabla\cdot \bm{D} & = \rho, \label{eq:maxwell1} \\
    \nabla \times \bm{E} +  \frac{\partial \bm{B}}{\partial t} & = \bm{0}, \label{eq:maxwell2} \\
    \nabla\cdot \bm{B} & = 0, \label{eq:maxwell3} \\
    \nabla \times \bm{H} - \frac{\partial \bm{D}}{\partial t} &= \bm{J}. \label{eq:maxwell4}
    \end{align}
\end{subequations}

Equation~\eqref{eq:maxwell} is automatically labeled by \texttt{cleveref},
as well as Equation~\eqref{eq:maxwell1} and Equation~\eqref{eq:maxwell3}.
Thanks to the \verb|cleveref| package, there is no need to use \verb|\eqref|.
Remember that Equations have to be numbered only if they are referenced in the text.

Equations~\eqref{eq:maxwell_multilabels1}, \eqref{eq:maxwell_multilabels2}, \eqref{eq:maxwell_multilabels3}, and \eqref{eq:maxwell_multilabels4} show again Maxwell's equations without brace:
\begin{align}
    \nabla\cdot \bm{D} & = \rho, \label{eq:maxwell_multilabels1} \\
    \nabla \times \bm{E} +  \frac{\partial \bm{B}}{\partial t} &= \bm{0}, \label{eq:maxwell_multilabels2} \\
    \nabla\cdot \bm{B} & = 0, \label{eq:maxwell_multilabels3} \\
    \nabla \times \bm{H} - \frac{\partial \bm{D}}{\partial t} &= \bm{J} \label{eq:maxwell_multilabels4}.
\end{align}

Equation~\eqref{eq:maxwell_singlelabel} is the same as before,
but with just one label:
\begin{equation}
    \label{eq:maxwell_singlelabel}
    \left\{
    \begin{aligned}
    \nabla\cdot \bm{D} & = \rho, \\
    \nabla \times \bm{E} +  \frac{\partial \bm{B}}{\partial t} &= \bm{0},\\
    \nabla\cdot \bm{B} & = 0, \\
    \nabla \times \bm{H} - \frac{\partial \bm{D}}{\partial t} &= \bm{J}.
    \end{aligned}
    \right.
\end{equation}

\section{Figures, Tables and Algorithms}
Figures, Tables and Algorithms have to contain a Caption that describe their content, and have to be properly reffered in the text.

\subsection{Figures}
\label{subsec:figures}

For including pictures in your text you can use \texttt{TikZ} for high-quality hand-made figures,
or just include them as usual with the command
\begin{verbatim}
\includegraphics[options]{filename.xxx}
\end{verbatim}
Here xxx is the correct format, e.g. \verb|.png|, \verb|.jpg|, \verb|.eps|, \dots.

\begin{figure}[H]
    \centering
    \includegraphics[width=0.3\textwidth]{logo_polimi_scritta.eps}
    \caption{Caption of the Figure to appear in the List of Figures.}
    \label{fig:quadtree}
\end{figure}

Thanks to the \texttt{\textbackslash subfloat} command, a single figure, such as Figure~\ref{fig:quadtree},
can contain multiple sub-figures with their own caption and label, e.g. \color{black} Figure~\ref{fig:polimi_logo1} and Figure~\ref{fig:polimi_logo2}. 

\begin{figure}[H]
    \centering
    \subfloat[One PoliMi logo.\label{fig:polimi_logo1}]{
        \includegraphics[scale=0.5]{Images/logo_polimi_scritta.eps}
    }
    \quad
    \subfloat[Another one PoliMi logo.\label{fig:polimi_logo2}]{
        \includegraphics[scale=0.5]{Images/logo_polimi_scritta2.eps}
    }
    \caption[Shorter caption]{This is a very long caption you don't want to appear in the List of Figures.}
    \label{fig:quadtree2}
\end{figure}


\subsection{Tables}
\label{subsec:tables}

Within the environments \texttt{table} and  \texttt{tabular} you can create very fancy tables as the one shown in Table~\ref{table:example}.
\begin{table}[H]
    \caption*{\textbf{Title of Table (optional)}}
    \centering 
    \begin{tabular}{|p{3em} c c c |}
    \hline
    \rowcolor{bluepoli!40} % comment this line to remove the color
     & \textbf{column 1} & \textbf{column 2} & \textbf{column 3} \T\B \\
    \hline \hline
    \textbf{row 1} & 1 & 2 & 3 \T\B \\
    \textbf{row 2} & $\alpha$ & $\beta$ & $\gamma$ \T\B\\
    \textbf{row 3} & alpha & beta & gamma \B\\
    \hline
    \end{tabular}
    \\[10pt]
    \caption{Caption of the Table to appear in the List of Tables.}
    \label{table:example}
\end{table}

You can also consider to highlight selected columns or rows in order to make tables more readable.
Moreover, with the use of \texttt{table*} and the option \texttt{bp} it is possible to align them at the bottom of the page. One example is presented in Table~\ref{table:exampleC}. 

\begin{table}[H]
\centering 
    \begin{tabular}{|p{3em} | c | c | c | c | c | c|}
    \hline
%    \rowcolor{bluepoli!40}
     & \textbf{column1} & \textbf{column2} & \textbf{column3} & \textbf{column4} & \textbf{column5} & \textbf{column6} \T\B \\
    \hline \hline
    \textbf{row1} & 1 & 2 & 3 & 4 & 5 & 6 \T\B\\
    \textbf{row2} & a & b & c & d & e & f \T\B\\
    \textbf{row3} & $\alpha$ & $\beta$ & $\gamma$ & $\delta$ & $\phi$ & $\omega$ \T\B\\
    \textbf{row4} & alpha & beta & gamma & delta & phi & omega \B\\
    \hline
    \end{tabular}
    \\[10pt]
    \caption{Highlighting the columns}
    \label{table:exampleC}
\end{table}

\begin{table}[H]
\centering 
    \begin{tabular}{|p{3em} c c c c c c|}
    \hline
%    \rowcolor{bluepoli!40}
     & \textbf{column1} & \textbf{column2} & \textbf{column3} & \textbf{column4} & \textbf{column5} & \textbf{column6} \T\B \\
    \hline \hline
    \textbf{row1} & 1 & 2 & 3 & 4 & 5 & 6 \T\B\\
    \hline
    \textbf{row2} & a & b & c & d & e & f \T\B\\
    \hline
    \textbf{row3} & $\alpha$ & $\beta$ & $\gamma$ & $\delta$ & $\phi$ & $\omega$ \T\B\\
    \hline
    \textbf{row4} & alpha & beta & gamma & delta & phi & omega \B\\
    \hline
    \end{tabular}
    \\[10pt]
    \caption{Highlighting the rows}
    \label{table:exampleR}
\end{table}

\subsection{Algorithms}
\label{subsec:algorithms}

Pseudo-algorithms can be written in \LaTeX{} with the \texttt{algorithm} and \texttt{algorithmic} packages.
An example is shown in Algorithm~\ref{alg:var}.
\begin{algorithm}[H]
    \label{alg:example}
    \caption{Name of the Algorithm}
    \label{alg:var}
    \label{protocol1}
    \begin{algorithmic}[1]
    \STATE Initial instructions
    \FOR{$for-condition$}
    \STATE{Some instructions}
    \IF{$if-condition$}
    \STATE{Some other instructions}
    \ENDIF
    \ENDFOR
    \WHILE{$while-condition$}
    \STATE{Some further instructions}
    \ENDWHILE
    \STATE Final instructions
    \end{algorithmic}
\end{algorithm} 

\vspace{5mm}

\section{Theorems, propositions and lists}

\subsection{Theorems}
Theorems have to be formatted as:
\begin{theorem}
\label{a_theorem}
Write here your theorem. 
\end{theorem}
\textit{Proof.} If useful you can report here the proof.

\subsection{Propositions}
Propositions have to be formatted as:
\begin{proposition}
Write here your proposition.
\end{proposition}

\subsection{Lists}
How to  insert itemized lists:
\begin{itemize}
    \item first item;
    \item second item.
\end{itemize}
How to insert numbered lists:
\begin{enumerate}
    \item first item;
    \item second item.
\end{enumerate}

%-------------------------------------------------------------------------
%	APPENDICES
%-------------------------------------------------------------------------

\cleardoublepage
\addtocontents{toc}{\vspace{2em}} % Add a gap in the Contents, for aesthetics
\appendix
\chapter{Appendix A}
If you need to include an appendix to support the research in your thesis, you can place it at the end of the manuscript.
An appendix contains supplementary material (figures, tables, data, codes, mathematical proofs, surveys, \dots)
which supplement the main results contained in the previous chapters.


% LIST OF FIGURES
\listoffigures

% LIST OF TABLES
\listoftables

\cleardoublepage

\end{document}
